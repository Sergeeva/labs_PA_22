\input{preamble}

\newcommand{\subject}{Параллельные алгоритмы}
\newcommand{\labnumber}{5}
\newcommand{\teacher}{Сергеева~Е.И.}
\newcommand{\theme}{Знакомство с программированием гетерогенных систем в стандарте Open CL}
\newcommand{\name}{Эйсвальд~М.И.}

\setlength{\extrarowheight}{1mm}

\begin{document}

\input{titlepage}

\setcounter{page}{2}

\subsection*{Цель работы.}
Получить навыки работы в стандарте OpenCL, изучить возможности стандарта OpenCL.

\subsection*{Выполнение работы.}

За основу работы было взято опубликованное в репозитории демонстрационное приложение, использующее OpenCL. Приложение засекает время, затраченное на нахождение приближения множества Мандельброта на заданном количестве пикселей с заданным числом итераций, после чего находит приближение множества Мандельброта на каждом устройстве компьютера, поддерживающем стандарт OpenCL, также засекая затраченное время. Результат вычислений с помощью OpenCL сохраняется в виде PNG-изображения с помощью сторонней библиотеки. Пример изображения представлен на рисунке ниже.

\begin{figure}[!h]
	\centering
	\includegraphics[width=\linewidth]{output}
	\caption{Визуализация приближения множества Мандельброта}
\end{figure}

Сравнение производительности последовательных вычислений и вычислений с помощью OpenCL представлено в таблице ниже.

\begin{table}[!h]
	\centering
	\begin{tabulary}{\textwidth}{|C|C|C|C|}
		\hline
		Размер изображения, пикселей&Количество итераций&Время работы итеративного алгоритма, мс&Время работы алгоритма на основе OpenCL, мс\\\hline
		680$\times$480&10&13&1\\\hline
		680$\times$480&50&34&2\\\hline
		680$\times$480&100&58&2\\\hline
		680$\times$480&1000&479&13\\\hline
		680$\times$480&10000&4640&106\\\hline
		800$\times$600&10&19&2\\\hline
		800$\times$600&50&50&3\\\hline
		800$\times$600&100&86&3\\\hline
		800$\times$600&1000&708&22\\\hline
		800$\times$600&10000&6789&151\\\hline
	\end{tabulary}
\end{table}

\subsection*{Выводы.}
В ходе выполнения работы были изучены основы работы со стандартом OpenCL. Было установлено, что скорость работы OpenCL-приложений над хорошо распараллеливаемой задачей на порядки превышает скорость работы итеративных решений. Результатом работы стало приложение, визуализирующее фрактал Мандельброта.

%\newpage
%\section{Демонстрация процесса создания цифровой подписи ECDSA}

\end{document}
